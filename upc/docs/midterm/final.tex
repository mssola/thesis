% Copyright (C) 2014-2020 Miquel Sabaté Solà <mikisabate@gmail.com>
%
% This program is free software: you can redistribute it and/or modify
% it under the terms of the GNU General Public License as published by
% the Free Software Foundation, either version 3 of the License, or
% (at your option) any later version.
%
% This program is distributed in the hope that it will be useful,
% but WITHOUT ANY WARRANTY; without even the implied warranty of
% MERCHANTABILITY or FITNESS FOR A PARTICULAR PURPOSE.  See the
% GNU General Public License for more details.
%
% You should have received a copy of the GNU General Public License
% along with this program.  If not, see <http://www.gnu.org/licenses/>.

\documentclass[a4paper,12pt]{article}

\usepackage{indentfirst}
\usepackage{makeidx}
\usepackage[pdftex]{graphicx}
\usepackage{wrapfig}
\usepackage[utf8]{inputenc}
\usepackage[english]{babel}
\usepackage{url}


%%
% New commands.

% Add an extra vertical space.
\newcommand{\espai}{\par\vspace{5mm}}

% How I do lists.
\newcommand{\mylist}{
\begin{itemize}
\setlength{\itemsep}{1pt}
\setlength{\parskip}{0pt}
\setlength{\parsep}{0pt}}
\newcommand{\mylistend}{\end{itemize}}

% Needed for the title.
\newcommand{\HRule}{\rule{\linewidth}{0.5mm}}

% So the links in the Table of Contents actually work.
\usepackage{hyperref}
\hypersetup{
  colorlinks,
  citecolor=black,
  filecolor=black,
  linkcolor=black,
  urlcolor=black
}


%%
% Hey! Ho! Let's go!


\begin{document}

% Title
% Copyright (C) 2014-2020 Miquel Sabaté Solà <mikisabate@gmail.com>
%
% This program is free software: you can redistribute it and/or modify
% it under the terms of the GNU General Public License as published by
% the Free Software Foundation, either version 3 of the License, or
% (at your option) any later version.
%
% This program is distributed in the hope that it will be useful,
% but WITHOUT ANY WARRANTY; without even the implied warranty of
% MERCHANTABILITY or FITNESS FOR A PARTICULAR PURPOSE.  See the
% GNU General Public License for more details.
%
% You should have received a copy of the GNU General Public License
% along with this program.  If not, see <http://www.gnu.org/licenses/>.


%%
% This file defines the title. It's created by hand by using the
% the instructions that were given in the following wiki page:
%       http://en.wikibooks.org/wiki/LaTeX/Title_Creation

\begin{titlepage}
\begin{center}

\textsc{\Large Barcelona School of Informatics}\\[0.5cm]

% Title
{ \small \HRule \\[0.4cm] }
{ \huge \bf Fita de seguiment \\[0.4cm] }
{ \small \HRule \\[0.4cm] }

% Author and supervisor
\begin{minipage}{0.4\textwidth}
\begin{flushleft} \large
\emph{Author:}\\
Miquel \textsc{Sabaté Solà}
\end{flushleft}
\end{minipage}
\begin{minipage}{0.4\textwidth}
\begin{flushright} \large
\emph{Director:} \\
Jordi \textsc{Garcia Almiñana}
\end{flushright}
\end{minipage}

\vfill

% Bottom of the page
{\large \today}

\end{center}
\end{titlepage}


\thispagestyle{empty}
\tableofcontents{\newpage}

\setcounter{page}{1}

\section{Context}

\subsection{The problem}

Cities around the world can generate a lot of data. This data is usually
fetched through sensors. Why do these cities fetch all this data? Basically,
each city can collect and process all this data and convert it into valuable
information. This information can then be used to improve the city in multiple
ways. Right know, there is an european effort called {\bf iCity} that groups
four european cities: London, Barcelona, Genoa and Bologna. These four cities
have multiple sensors on their streets that collect a wide variety of data:

\mylist
  \item Air pollution.
  \item Traffic.
  \item Irrigation control.
  \item Pedestrian flow.
\mylistend

\subsection{The idea}

With all this quantity of data going on, we need a reliable platform that
integrates everything in an easy way. The {\bf iCity} platform has shown to be
quite good at integrating all the cities. However, this platform is too raw: it
just contains the fetched data as-is. It would be more useful if some platform
could wrap the data from {\bf iCity} and provide more rich and useful services
and information around it.

My Bachelor's Degree Thesis is about building this platform. This platform will
be built with the Storm framework. This framework will allow us to process all
the data from {\bf iCity} and provide rich information in {\bf realtime}.

\subsection{Similar projects}

As part of my Bachelor Degree Thesis, I have done some research on this area. I
have not found any project that has an identical aim as this platform. However,
I have to admit that this project overlaps a bit with the iCity platform.

\subsection{Challenges}

It has passed some months now since I started this project. When I started I
hoped to obtain all the information also from the Barcelona OpenData
iniciative. However, this iniciative turned out to be not what I expected: it
is just a repository of static data. Since my platform requires dynamic and
realtime data, fetching static data is pointlesss. This situation did not
affect this project greatly, since the iCity platform turned out to be exactly
what I needed, and, therefore, I did not care about Barcelona's OpenData any
longer.

Another challenge has been that I am doing this project with the Scala
programming language. To be honest, I could have done this project with just
Java, since Storm is available to any programming language that runs on the
JVM. Scala, however, is more modern than Java, and I am having more fun writing
Scala code than with Java. So, even if I am new to Scala and it is a challenge,
it is a challenge in which I am having a lot of fun.

\section{Scheduling}

\subsection{Changes on my original schedule}

There have not been any major change to my initial schedule. I have to admit
that for personal reasons I have started the project a couple of weeks later
than expected. This means that my final schedule has shifted a bit from my
original one. This has not resulted into many problems because:

\mylist
  \item In my orignal Gantt I kept the ``Analysis and design'' and the
``Development of the core'' phases seperately. I finally merged these two
phases into one. This turned out to be fine.
  \item The ``Providing services'' phase has been shorter than expected.
\mylistend

I have not re-adjusted my methodology of work in order to keep the schedule.
Moreover, since there are not many alternatives out there in this field, I
have not spent too much time investigating on other possible designs or
implementations.

\subsection{State of my thesis}

Taking into consideration the fact that I have shifted the schedule a couple of
weeks, I can conclude that I am on schedule. More specifically, this is what I
have right now:

\mylist
  \item A couple of services have already been roughly implemented. I expect to
fully test them in the following weeks.
  \item I have made some calculations around the requirements of the system.
These numbers will be more clear in the following weeks, when I will make some
benchmarks.
\mylistend

Therefore, I have already done all the heavy lifting of my thesis. Now, this is
what is still left to be done:

\mylist
  \item Improve the current services. I have not document them yet and some
tests can be improved.
  \item I have not done any real benchmark yet. Therefore, any calculation
that I have made until now about the requirements of the platform is still an
approximation.
\mylistend

\section{Laws and regulation}

The platform that I am building in my Thesis fetches all the data from {\bf
iCity}. The iCity platform is an european effort that fetches all the data in a
safe and lawful way. Therefore, I conclude that my platform is not, in any way,
violating any european law.

\section{Knowledge learnt at this university}

Lastly, I would like to say that this project has served me greatly to apply
all the knowledge that I have accumulated during my time at this university.
More specifically:

\mylist
  \item {\bf Software design}: I have learnt a lot on designing software at this
university. In this project I had the oportunity to apply all this knowledge in
order to design and build the platform itself.
  \item {\bf Clusters and distributed architectures}: I have taken some courses
regarding clusters and distributed architectures. These course have proved me
how important they were.
\mylistend

\end{document}
