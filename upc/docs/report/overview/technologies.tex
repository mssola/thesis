 
\section{Used Technologies}

\subsection{Linux}

This platform targets Linux\footnote{http://www.linux.com/}. It is the Operating
System that I have used to develop this platform and the target of this
platform. This is because, for a variety of reasons, the vast majority of
Internet servers use Linux as their Operating system. Therefore, it makes
perfect sense to use Linux as the target operating system.

Note that this platform should also work in MacOS X and other \acs{BSD}
variants. However, I have not tested this platform in these operating systems,
so I cannot claim anything in regards to availability here.

\subsection{Java \& Scala}

Storm is implemented with Java and Clojure\footnote{http://clojure.org/}. This
two languages sit in top of
the \ac{JVM}. The \ac{JVM} is a virtual machine that can be used by any language
that knows how to produce bytecode for it. This includes languages such as:
Java, Clojure, Scala, Groovy, etc.

Moreover, all the languages on top of the \ac{JVM} can share packages. That is,
a \acs{JAR} file can be used as a library from any of these languages,
regardless of the language that was used in the library inside the \acs{JAR}
file. This means that languages like Scala can re-use the huge list of Java
packages that the community has implemented.

This means that in order to implement this platform I could have used any
language that sits on top of the \ac{JVM} without any real problem. I have
chosen Scala\cite{scala} because:

\mylist
  \item It is as {\bf fast} as Java, so there is no performance penalties
because of the language if we compare it with Java.
  \item It is {\bf modern}. Scala is a more recent language. This means that it
has had the influence of languages that are more recent than Java. This results
in Scala having many concepts from functional programming languages, concepts
from Python, Ruby, etc. It is an absolute pleasure to write Scala code.
  \item It has a {\bf robust} approach of concurrency. It is far more intuitive
than Java's lock/unlock mechanisms.
  \item It is {\bf stable}. Even if it is more recent than Java, Scala is rock
solid. As an example, big companies like Twitter and Foursquare have lots of
Scala code running on production.
\mylistend

\subsection{Storm}

As I have said in the section \ref{sec:state_storm}, the technology that I am
using to process all the data is Storm.

\subsection{Cassandra}
\label{sec:cassandra}

I do not store a lot of data in this platform, but I do store some data like
the state of the cluster. To achieve this I use Apache
Cassandra\cite{cassandra}. We could have chosen any \ac{DBMS} here to do the
job, but I have chosen Cassandra for the following reasons:

\mylist
  \item It has support for MapReduce. This is the main reason that I have
chosen Cassandra instead any other traditional \ac{DBMS} like PostgreSQL.
  \item It is {\bf fault-tolerant}, and supports several replication policies
across multiple clusters.
  \item It is {\bf descentralized}. There is no single point of failure.
  \item Its NoSQL nature has been helpful throughout the development of this
platform. Moreover, its query language (CQL) is quite similar to standard SQL,
so there has not been any major learning curve for me to use Cassandra.
\mylistend

One could argue that we do not need any \ac{DBMS} at all: we could just write
into temporary files or something like that. Even though would have simplified
things, this is not possible for the following reasons:

\begin{enumerate}
  \itemsep0em
  \item This platform can run in a multi-node cluster. Therefore, we would have
to pick a node among the others to store these temporary files. Doing this is,
of course, undesirable.
  \item Even if we pick a node from the cluster to keep these files: what if
this node melts down ? We cannot risk the integrity of the entire cluster to a
node that stores temporary files.
  \item CQL is a nice SQL-like language that gives us more flexibility doing
some operations on files.
\end{enumerate}

\subsection{Go}
\label{sec:go}

Go\footnote{http://golang.org/} is a programming language that emphasizes
concurrent programming. It was first developed by Google but it is now a truly
open source project. I use Go in the API layer. I have chosen Go for the
following reasons:

\begin{enumerate}
  \item Go puts special emphasis on {\bf concurrency}. Concurrency is also a
big deal in the Storm application. Therefore, it is a perfect match to use a
language that deals with concurrency and parallelism in an elegant and powerful
way.
  \item Go is a {\bf compiled} language. Therefore it is fast.
  \item It is {\bf simple}. As you can see in the language
reference\footnote{http://golang.org/ref/spec}, this language is pragmatic and
simple. Plus, programmers comming from languages like C and C++ will find a lot
of similarities with this language.
  \item It is {\bf stable}. Even if it is a fairly recent language, it is rock
solid. For example, companies like Google, Youtube, Dropbox, Soundcloud, etc.
are already using Go in production.
  \item The {\bf net/http} package is simple but really powerful. It abstracts
away a lot of problems regarding HTTP requests and responses.
\end{enumerate}
