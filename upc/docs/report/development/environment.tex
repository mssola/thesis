
\section{Environmental impact}

In the same way that this platform can bring a lot of goodness in the social
front, it certainly comes with a cost. In my case the cost is an environmental
impact that cannot be understated.

This platform has to run in a cluster. A cluster, by itself, presents a lot of
environmental issues that have to be dealt. If we do not do that, and we do not
care about the environmental impact, it might happen that the cluster that
executes this platform might be really aggressive towards the environment.

Another way to see it is on the bills. If, for example, this cluster does not
make a proper use of electricity, bills will be higher.

Regardless how you see it, taking care of the environmental impact is a vital
aspect of the implementation of this platform. The components that are more
damaging to the environment are the following:

\mylist
  \item Power supply.
  \item Maintaining a cooling system.
  \item The implied environmental costs of building the cluster.
\mylistend

All of these can be reduced by using the minimum amount of cluster time as
possible. This means to run the software in ``batches'' or with a very low
latency. However, this is not possible at all if the cluster has a lot of
requests, and that is to be expected.

So, the only way to reduce the environmental impact is to build the cluster
with components that consume as least as possible. The components integrated in
the cluster have to have strong policies regarding environmental issues.
Therefore, in this project there is not much that I can do to reduce the
environmental impact: it only depends on the components of the cluster.
