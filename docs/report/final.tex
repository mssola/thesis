% Copyright (C) 2014 Miquel Sabaté Solà <mikisabate@gmail.com>
% This file is licensed under the MIT license.
% See the LICENSE file.

\documentclass[a4paper,12pt]{article}

\usepackage{indentfirst}
\usepackage{makeidx}
\usepackage[pdftex]{graphicx}
\usepackage{wrapfig}
\usepackage[utf8]{inputenc}
\usepackage[english]{babel}
\usepackage{url}


%%
% New commands.

% Add an extra vertical space.
\newcommand{\espai}{\par\vspace{5mm}}

% How I do lists.
\newcommand{\mylist}{
\begin{itemize}
\setlength{\itemsep}{1pt}
\setlength{\parskip}{0pt}
\setlength{\parsep}{0pt}}
\newcommand{\mylistend}{\end{itemize}}

% Needed for the title.
\newcommand{\HRule}{\rule{\linewidth}{0.5mm}}

% So the links in the Table of Contents actually work.
\usepackage{hyperref}
\hypersetup{
  colorlinks,
  citecolor=black,
  filecolor=black,
  linkcolor=black,
  urlcolor=black
}


%%
% Hey! Ho! Let's go!


\begin{document}

% Title
% Copyright (C) 2014-2020 Miquel Sabaté Solà <mikisabate@gmail.com>
%
% This program is free software: you can redistribute it and/or modify
% it under the terms of the GNU General Public License as published by
% the Free Software Foundation, either version 3 of the License, or
% (at your option) any later version.
%
% This program is distributed in the hope that it will be useful,
% but WITHOUT ANY WARRANTY; without even the implied warranty of
% MERCHANTABILITY or FITNESS FOR A PARTICULAR PURPOSE.  See the
% GNU General Public License for more details.
%
% You should have received a copy of the GNU General Public License
% along with this program.  If not, see <http://www.gnu.org/licenses/>.


%%
% This file defines the title. It's created by hand by using the
% the instructions that were given in the following wiki page:
%       http://en.wikibooks.org/wiki/LaTeX/Title_Creation

\begin{titlepage}
\begin{center}

\textsc{\Large Barcelona School of Informatics}\\[0.5cm]

% Title
{ \small \HRule \\[0.4cm] }
{ \huge \bf Fita de seguiment \\[0.4cm] }
{ \small \HRule \\[0.4cm] }

% Author and supervisor
\begin{minipage}{0.4\textwidth}
\begin{flushleft} \large
\emph{Author:}\\
Miquel \textsc{Sabaté Solà}
\end{flushleft}
\end{minipage}
\begin{minipage}{0.4\textwidth}
\begin{flushright} \large
\emph{Director:} \\
Jordi \textsc{Garcia Almiñana}
\end{flushright}
\end{minipage}

\vfill

% Bottom of the page
{\large \today}

\end{center}
\end{titlepage}


\tableofcontents{\newpage}
\thispagestyle{empty}

\setcounter{page}{1}

\section{Preface}

Això serà un apartat {\bf molt} curt que bàsicament anirà de com està la
situació actual de la tecnologia, i com està afectant al desenvolupament de les
ciutats.

Ja et dic, una cosa exageradament curta (una pàgina màxima). Serveix per situar
al lector de la meva motivació personal per fer el projecte. També donaria els
típics agraïments.

\section{Project}

Aquest és l'apartat on explicaré tot allò que envolta al projecte. Començaré
parlant del context d'aquesta plataforma (apartat 2.1): Quines tecnologies hi
ha, quin entorn estem vivint, etc.

Després de presentar el context, parlaré de quin problema intenta resoldre
aquest projecte (apartat 2.2). I com en tot problema, s'ha de donar una idea,
una solució (apartat 2.3).

Després parlaré de coses com projectes similars (apartat 2.4) que existeixen,
alternatives, etc. També parlaré dels reptes que m'he trobat creant aquesta
aplicació (apartat 2.5).

En resum, aquest apartat serveix per vendre el projecte. No entra massa en
detall tecnològicament parlant, però presenta quin problema tenim, i com ho
intentem solucionar.

El format d'aquesta secció és molt similar al que he anat fent tant en GEP com
en la fita de seguiment.

\subsection{The Context}
\subsection{The Problem}
\subsection{The Idea}
\subsection{Similar projects}
\subsection{Challenges}

\section{Implementation}

Aquesta serà la secció més ``dura'' del projecte. Tractarè dels detalls més
tècnics del projecte. Començaré parlant de ``L'estat de l'art'' (apartat 3.1),
que va de com està la tecnologia actualment al voltant del projecte.
Plataformes, llenguatges, etc. La conclusió d'aquest apartat estarà en el
següent apartat (3.1), que tractarà sobre quines tecnologies he agafat i perquè
ho he fet.

Una vegada parlat de les tecnologies escollides, passaré a comentar la
implementació de la plataforma, del software (apartat 3.3). Com que això pot
ser realment dur, ho intentaré dividir en subseccions més petites. Aquí
intentaré entrall més en detall. No ficaré tot el codi, ja que el codi el
posaré disponible a Github i ficaré un link per a que qualsevol ho pugui veure.
En aquest apartat més aviat parlaré del disseny del software, i ficaré les
porcions de codi més crítiques.

Finalment això ho remato parlant del hardware (apartat 3.4). Sobre el hardware
comentaré les especificacions mínimes, les òptimes que ha de tenir el cluster
que executi aquest programa.

Intentaré fer benchmarks (amb el lab. de càlcul del DAC), benchmarks locals,
etc. Aquest serà l'apartat 3.5.

\subsection{State of the Art}
\subsection{Chosen technologies}
\subsection{The Platform}
\subsection{The Hardware}
\subsection{Running this Platform}

\section{The Project}

Aquesta secció sincerament no sé si ficar-la abans o després de la secció
d'Implementació :/ En primera instància parlaré de la planificació del
projecte: la planificació original, com ha canviat, i finalment com ha sigut.
Es tracta de explicar com ha evolucionat tot plegat.

Finalemnt parlaré del pressupost. Intentaré ser bastant més exhaustiu i
considerar més coses que quan vaig fer-ho a GEP.

\subsection{The Schedule}
\subsection{The Budget}
\subsection{Laws and Regulation}

\section{Conclusions}

Aquesta és la secció típica de conclusions. Com que encara estic a mitges, no
sé si això serà molt llarg o que. Si és molt llarg, el més segur és que acabi
dividint-ho en més subseccions. Ja veurem. (+ el que he après en aquesta
universitat)

\section{Bibliography}

Finalment ficaré l'apartat de bibliografia del projecte.

\end{document}
