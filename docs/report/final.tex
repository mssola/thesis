% Copyright (C) 2014 Miquel Sabaté Solà <mikisabate@gmail.com>
% This file is licensed under the MIT license.
% See the LICENSE file.

\documentclass[a4paper,12pt]{article}

\usepackage{indentfirst}
\usepackage{makeidx}
\usepackage[pdftex]{graphicx}
\usepackage{wrapfig}
\usepackage[utf8]{inputenc}
\usepackage[english]{babel}
\usepackage{url}


%%
% New commands.

% Add an extra vertical space.
\newcommand{\espai}{\par\vspace{5mm}}

% How I do lists.
\newcommand{\mylist}{
\begin{itemize}
\setlength{\itemsep}{1pt}
\setlength{\parskip}{0pt}
\setlength{\parsep}{0pt}}
\newcommand{\mylistend}{\end{itemize}}

% Needed for the title.
\newcommand{\HRule}{\rule{\linewidth}{0.5mm}}

% So the links in the Table of Contents actually work.
\usepackage{hyperref}
\hypersetup{
  colorlinks,
  citecolor=black,
  filecolor=black,
  linkcolor=black,
  urlcolor=black
}


%%
% Hey! Ho! Let's go!


\begin{document}

% Title
% Copyright (C) 2020 Miquel Sabaté Solà <mikisabate@gmail.com>
%
% This program is free software: you can redistribute it and/or modify
% it under the terms of the GNU General Public License as published by
% the Free Software Foundation, either version 3 of the License, or
% (at your option) any later version.
%
% This program is distributed in the hope that it will be useful,
% but WITHOUT ANY WARRANTY; without even the implied warranty of
% MERCHANTABILITY or FITNESS FOR A PARTICULAR PURPOSE.  See the
% GNU General Public License for more details.
%
% You should have received a copy of the GNU General Public License
% along with this program.  If not, see <http://www.gnu.org/licenses/>.

%%
% Based on the one I did for the UPC.

\begin{titlepage}

\begin{center}

\textsc{\Large Universitat Oberta de Catalunya}\\[0.5cm]
\textsc{Grau de Llengua i Literatura Catalanes}\\[0.5cm]
\textsc{\small Treball de fi de grau}\\[0.5cm]

% Title
{ \small \HRule \\[0.4cm] }
{ \huge \bf Per decidir \\[0.4cm] }
{ \small \HRule \\[0.4cm] }

% Author and supervisor
\begin{minipage}{0.4\textwidth}
\begin{flushleft} \large
\emph{Autor:}\\
Miquel \textsc{Sabaté Solà}
\end{flushleft}
\end{minipage}
\begin{minipage}{0.4\textwidth}
\begin{flushright} \large
\emph{Director:} \\
Per \textsc{Decidir}
\end{flushright}
\end{minipage}

\vfill

% Bottom of the page
{\large Capellades, \today}

\end{center}
\end{adjustwidth*}
\end{titlepage}

\tableofcontents{\newpage}
\thispagestyle{empty}

\setcounter{page}{1}

\section{Preface}

Això serà un apartat {\bf molt} curt que bàsicament anirà de com està la
situació actual de la tecnologia, i com està afectant al desenvolupament de les
ciutats.

Ja et dic, una cosa exageradament curta (una pàgina màxima). Serveix per situar
al lector de la meva motivació personal per fer el projecte. També donaria els
típics agraïments.

\section{Project}

Aquest és l'apartat on explicaré tot allò que envolta al projecte. Començaré
parlant del context d'aquesta plataforma (apartat 2.1): Quines tecnologies hi
ha, quin entorn estem vivint, etc.

Després de presentar el context, parlaré de quin problema intenta resoldre
aquest projecte (apartat 2.2). I com en tot problema, s'ha de donar una idea,
una solució (apartat 2.3).

Després parlaré de coses com projectes similars (apartat 2.4) que existeixen,
alternatives, etc. També parlaré dels reptes que m'he trobat creant aquesta
aplicació (apartat 2.5).

En resum, aquest apartat serveix per vendre el projecte. No entra massa en
detall tecnològicament parlant, però presenta quin problema tenim, i com ho
intentem solucionar.

El format d'aquesta secció és molt similar al que he anat fent tant en GEP com
en la fita de seguiment.

\subsection{The Context}
\subsection{The Problem}
\subsection{The Idea}
\subsection{Similar projects}
\subsection{Challenges}

\section{Implementation}

Aquesta serà la secció més ``dura'' del projecte. Tractarè dels detalls més
tècnics del projecte. Començaré parlant de ``L'estat de l'art'' (apartat 3.1),
que va de com està la tecnologia actualment al voltant del projecte.
Plataformes, llenguatges, etc. La conclusió d'aquest apartat estarà en el
següent apartat (3.1), que tractarà sobre quines tecnologies he agafat i perquè
ho he fet.

Una vegada parlat de les tecnologies escollides, passaré a comentar la
implementació de la plataforma, del software (apartat 3.3). Com que això pot
ser realment dur, ho intentaré dividir en subseccions més petites. Aquí
intentaré entrall més en detall. No ficaré tot el codi, ja que el codi el
posaré disponible a Github i ficaré un link per a que qualsevol ho pugui veure.
En aquest apartat més aviat parlaré del disseny del software, i ficaré les
porcions de codi més crítiques.

Finalment això ho remato parlant del hardware (apartat 3.4). Sobre el hardware
comentaré les especificacions mínimes, les òptimes que ha de tenir el cluster
que executi aquest programa.

Intentaré fer benchmarks (amb el lab. de càlcul del DAC), benchmarks locals,
etc. Aquest serà l'apartat 3.5.

\subsection{State of the Art}
\subsection{Chosen technologies}
\subsection{The Platform}
\subsection{The Hardware}
\subsection{Running this Platform}

% Section where I talk about what I needed to do in order to complete my
% thesis. Nothing too fancy...

\section{The Project}

% We first talk about the schedule.
 
\section{The Schedule}

\subsection{Initial planning}

I divided the load of work of my project into four clear stages. The first one
was dedicated to study the feasability of the project and to do the initial
planning of the project. This was done through a course called \ac{GEP}. This
course includes the following stages:

\mylist
  \item Scope of the project.
  \item Project planning.
  \item Budget and sustainability.
  \item Preliminary presentation.
  \item Bibliography.
  \item List of conditions.
  \item Oral presentation and delivery of the final document.
\mylistend

\subsubsection*{Project analysis and design}

After taking the \ac{GEP} course, I moved on and started the stage of ``Project
analysis and design''. The main goal of this stage was to draw a clear picture
of the project and analyze all the goals of the project.

Therefore, this stage was made out of two sub stages. The first one, the
analysis of the project. In this sub stage I devoted myself to define all
the goals, requirements, features and use cases of my plaform. The other sub
stage consisted in design of the platform itself.

\subsubsection*{Project iterations}

\begin{enumerate}
\item {\bf Development of the core software infrastructure}

In this iteration I focused on the core infrastructure that has to
hold the whole platform. This was executed only in the software front.

\item {\bf Providing services}

The next iteration consisted on building the services on top of the base
infrastructure that had been created in the previous stage.

\item {\bf Designing the cluster}

At this point, we had the software ready to be deploy in production. So in this
step I focused on writing down the specifications of an ideal cluster.

\item {\bf Concluding the development}

In the last iteration I concluded the development of this project: running
tests, final checks of the code, etc.
\end{enumerate}


\subsubsection*{Final stage}

The final stage consisted on closing the project. The main points of this stage
were:

\mylist
  \item Write the documentation of the platform.
  \item Write the Final report (this document).
  \item Give the lecture of my thesis (June 27, 2014).
\mylistend

\subsubsection*{Gantt chart}

This initial planning was summarized in the following Gantt chart:

\begin{figure}[H]
  \hspace*{-2cm}
  \includegraphics[scale=0.75]{development/images/gantt.png}
  \caption{Gantt chart}\label{fig:gantt}
\end{figure}

One thing to note is that I computed my work load in weeks instead of computing
it with hours. This might be odd, but it proved to be clear and simple. In my
final report of the \ac{GEP} course I stated that the total amount of hours per
day was not fixed. However, I changed this policy to a much saner approach:

\mylist
  \item On {\bf weekdays} I dedicated 2 hours per day. This is because I had a
day to day job and I also took more courses at university. This meant that I
could not spent too much time during weekdays. Approximately, though, I
dedicated 2 hours per day.
  \item On the {\bf weekend} I dedicated approximately 8 hours to this project.
\mylistend

As you can see, I have been quite flexible regarding hours dedicated per day.
Anyways, in total I have spent 17 weeks developing this project . This means
that in total I have spent approximately:

\begin{center}
  17 weeks $\times$ (5 weekdays $\times$ 2 hours + 8 hours on the
weekend) $=$ 306 hours
\end{center}

\subsection{Changes in the midterm evaluation}

There have not been any major change to my initial schedule. I have to admit
that for personal reasons I started the project a couple of weeks later
than expected. This means that by my midterm evaluation (``Fita de
seguiment''), my final schedule shifted a bit from my original one. This did not
result into many problems because:

\mylist
  \item In my orignal Gantt I kept the ``Analysis and design'' and the
``Development of the core'' phases seperately. I merged these two phases into
one. This turned out to be fine.
  \item The ``Providing services'' phase was shorter than expected.
\mylistend

\subsection{Final changes}

Apart from my changes on the schedule by the time of my midterm evaluation, I
did not have to do anymore changes in my schedule. The changes on the schedule
did not affect negatively in the development of this project.

\subsection{Conclusions}

As you have seen, my initial schedule was surprisingly on point. I only had to
do some minor changes on the schedule by the times of the midterm evaluation
and that was it.

Even though I followed the schedule successfully, during the development of
this project I felt that I was working too much during the stages ``Development
of the core'' and ``Providing services''. I think that one of the weakest
points of my initial schedule was that I decided to spend too much time on
stages like ``Planning'' and ``Analysis and design''. I probably could have
shortened this two stages, so I had more time on other stages that required
more hard work.


% Yeah, right, I had a budget ...

\section{Budget}

\subsection{Introduction}

In this delivery I am focusing on the budget and on the impact of my project.
First of all, I am going to focus on the budget. This includes doing some math
in the following topics: human resources, hardware and software.

Lastly, I will be explaining the impact that I expect that this project will
have. My only considerations for the impact will be the social impact and the
environmental impact.

\subsection{Budget}

In this section I will be describing the estimated costs that this project will
have. After specifying each of the costs, I will make up a total amount of
estimated costs. This will lead me to conclude the needed budget for this
project.

\subsection{Human resources}

In my humble opinion the words ``human resources'' are quite awful, since it
feels like we are treating people like objects. Far from that, what I want to
do in this section is to describe all the costs related to employing people to
develop this project.

Luckily enough, it is just me in this project, so I just have to computemy own
salary. After doing some research, I have found that the average salary for a
Big data analyst is around 99,000 USD a year. This boils down to 51 USD per
hour. I did my scheduling based on weeks, instead of hours, and I think that I
will spend 20 hours in average per week. This means that in total, my salary
during the project will be:

\[
  20\ hours \cdot 16\ weeks \cdot 51\ USD = 16,320\ USD
\]


\subsection{Hardware}

Ideally in this section I would explain all the costs regarding the hardware.
However, I can't compute this because the only hardware that I will specify will
be the cluster which, in turn, is just a proposal. Other things like my laptop
won't be counted in this section They are not a cost for this project since I
use them for personal purposes too.

\subsection{Software}

In this section I will describe all the costs associated with software. In
short, this is my budget for software:

\[
  0\ \$
\]

All the software that I will be using for this project is free: both as in
``free beer'' as in ``free speech''. There are no fees for any license, I won't
be using any paid service, etc. Nothing.

\subsection{Total}

The main conclusion is that the only cost associated to this project is just my
salary. Therefore, my estimated costs are a total of \$16,320.

\subsection{Social impact}

This section is really important for this project. It is about the impact that
this platform will make to society. This is not a direct impact, but an
indirect one. The social impact of this platform comes in two ways:

\mylist
  \item The fact that a number of businesses will take advantage that this
platform exists.
  \item Indirectly, all the impact that all the services built on top of this
platform will produce.
\mylistend

This means that, ideally, the social impact of this platform can be huge.

\subsection{Environmental impact}

In the same way that this platform can bring a lot of goodness in the social
front, it certaintly comes with a cost. In my case the cost is an environmental
impact that cannot be understated.

In this project I will propose an ideal cluster that would be able to run the
software that I will design. The problem here is that maintaining a cluster
means the following:

\mylist
  \item Power supply.
  \item Maintaining a cooling system.
  \item The implied environmental costs of building the cluster.
\mylistend

All of these can be reduced by using the minimum amount of cluster time as
possible. This means to run the software in ``batches'' or with a very low
latency. However, this is not possible at all if the cluster has a lot of
requests, and that is to be expected.

Therefore, in the development of the cluster I will focus most of my efforts
into keeping the cluster as environmental friendly as possible.


% Additional sugar.
 
\subsection{BETA: Laws and Regulation}

The platform that has been built during my Bachelor Degree Thesis does not
violate any European or International law. From the very beginning it has been
stated that this platform fetches all the data from the {\it iCity} platform.
This platform is an European effort that has been created according to the laws
of the European Union. Therefore, the platform that I have built in my Thesis
is also bound by the laws of the European Union.

For this same reason, I conclude that this platform is not infringing in any
way the laws and the regulations from the European Union. Moreover, all the
infraestructures integrated in the {\it iCity} platform, have their own
policies and rules. This does not affect this project because all the
infraestructures that I have chosen are free to use and with no limitations
whatsoever.




\section{Conclusions}

Aquesta és la secció típica de conclusions. Com que encara estic a mitges, no
sé si això serà molt llarg o que. Si és molt llarg, el més segur és que acabi
dividint-ho en més subseccions. Ja veurem. (+ el que he après en aquesta
universitat)

\section{Bibliography}

Finalment ficaré l'apartat de bibliografia del projecte.

\end{document}
