% Copyright (C) 2014 Miquel Sabaté Solà <mikisabate@gmail.com>
% This file is licensed under the MIT license.
% See the LICENSE file.

\documentclass[a4paper,12pt]{article}

\usepackage{indentfirst}
\usepackage{makeidx}
\usepackage[pdftex]{graphicx}
\usepackage{wrapfig}
\usepackage[utf8]{inputenc}
\usepackage[english]{babel}
\usepackage{url}


%%
% New commands.

% Add an extra vertical space.
\newcommand{\espai}{\par\vspace{5mm}}

% How I do lists.
\newcommand{\mylist}{
\begin{itemize}
\setlength{\itemsep}{1pt}
\setlength{\parskip}{0pt}
\setlength{\parsep}{0pt}}
\newcommand{\mylistend}{\end{itemize}}

% Needed for the title.
\newcommand{\HRule}{\rule{\linewidth}{0.5mm}}

% So the links in the Table of Contents actually work.
\usepackage{hyperref}
\hypersetup{
  colorlinks,
  citecolor=black,
  filecolor=black,
  linkcolor=black,
  urlcolor=black
}


%%
% Hey! Ho! Let's go!


\begin{document}

% Title
% Copyright (C) 2014-2020 Miquel Sabaté Solà <mikisabate@gmail.com>
%
% This program is free software: you can redistribute it and/or modify
% it under the terms of the GNU General Public License as published by
% the Free Software Foundation, either version 3 of the License, or
% (at your option) any later version.
%
% This program is distributed in the hope that it will be useful,
% but WITHOUT ANY WARRANTY; without even the implied warranty of
% MERCHANTABILITY or FITNESS FOR A PARTICULAR PURPOSE.  See the
% GNU General Public License for more details.
%
% You should have received a copy of the GNU General Public License
% along with this program.  If not, see <http://www.gnu.org/licenses/>.


%%
% This file defines the title. It's created by hand by using the
% the instructions that were given in the following wiki page:
%       http://en.wikibooks.org/wiki/LaTeX/Title_Creation

\begin{titlepage}
\begin{center}

\textsc{\Large Barcelona School of Informatics}\\[0.5cm]

% Title
{ \small \HRule \\[0.4cm] }
{ \huge \bf Fita de seguiment \\[0.4cm] }
{ \small \HRule \\[0.4cm] }

% Author and supervisor
\begin{minipage}{0.4\textwidth}
\begin{flushleft} \large
\emph{Author:}\\
Miquel \textsc{Sabaté Solà}
\end{flushleft}
\end{minipage}
\begin{minipage}{0.4\textwidth}
\begin{flushright} \large
\emph{Director:} \\
Jordi \textsc{Garcia Almiñana}
\end{flushright}
\end{minipage}

\vfill

% Bottom of the page
{\large \today}

\end{center}
\end{titlepage}


\tableofcontents{\newpage}
\thispagestyle{empty}

\setcounter{page}{1}

\section{Preface}

Això serà un apartat {\bf molt} curt que bàsicament anirà de com està la
situació actual de la tecnologia, i com està afectant al desenvolupament de les
ciutats.

Ja et dic, una cosa exageradament curta (una pàgina màxima). Serveix per situar
al lector de la meva motivació personal per fer el projecte. També donaria els
típics agraïments.

\section{Project}

Aquest és l'apartat on explicaré tot allò que envolta al projecte. Començaré
parlant del context d'aquesta plataforma (apartat 2.1): Quines tecnologies hi
ha, quin entorn estem vivint, etc.

Després de presentar el context, parlaré de quin problema intenta resoldre
aquest projecte (apartat 2.2). I com en tot problema, s'ha de donar una idea,
una solució (apartat 2.3).

Després parlaré de coses com projectes similars (apartat 2.4) que existeixen,
alternatives, etc. També parlaré dels reptes que m'he trobat creant aquesta
aplicació (apartat 2.5).

En resum, aquest apartat serveix per vendre el projecte. No entra massa en
detall tecnològicament parlant, però presenta quin problema tenim, i com ho
intentem solucionar.

El format d'aquesta secció és molt similar al que he anat fent tant en GEP com
en la fita de seguiment.

\subsection{The Context}
\subsection{The Problem}
\subsection{The Idea}
\subsection{Similar projects}
\subsection{Challenges}

\section{Implementation}

Aquesta serà la secció més ``dura'' del projecte. Tractarè dels detalls més
tècnics del projecte. Començaré parlant de ``L'estat de l'art'' (apartat 3.1),
que va de com està la tecnologia actualment al voltant del projecte.
Plataformes, llenguatges, etc. La conclusió d'aquest apartat estarà en el
següent apartat (3.1), que tractarà sobre quines tecnologies he agafat i perquè
ho he fet.

Una vegada parlat de les tecnologies escollides, passaré a comentar la
implementació de la plataforma, del software (apartat 3.3). Com que això pot
ser realment dur, ho intentaré dividir en subseccions més petites. Aquí
intentaré entrall més en detall. No ficaré tot el codi, ja que el codi el
posaré disponible a Github i ficaré un link per a que qualsevol ho pugui veure.
En aquest apartat més aviat parlaré del disseny del software, i ficaré les
porcions de codi més crítiques.

Finalment això ho remato parlant del hardware (apartat 3.4). Sobre el hardware
comentaré les especificacions mínimes, les òptimes que ha de tenir el cluster
que executi aquest programa.

Intentaré fer benchmarks (amb el lab. de càlcul del DAC), benchmarks locals,
etc. Aquest serà l'apartat 3.5.

\subsection{State of the Art}
\subsection{Chosen technologies}
\subsection{The Platform}
\subsection{The Hardware}
\subsection{Running this Platform}

% Section where I talk about what I needed to do in order to complete my
% thesis. Nothing too fancy...

\section{TODO: The Project}

% We start by making a statement on the situation of the context.

\section{TODO: The Context}

Aquest és l'apartat on explicaré tot allò que envolta al projecte. Començaré
parlant del context d'aquesta plataforma (apartat 2.1): Quines tecnologies hi
ha, quin entorn estem vivint, etc.


% Then we talk about the problem we're trying to solve.

\subsection{TODO: The Problem}

Després de presentar el context, parlaré de quin problema intenta resoldre
aquest projecte (apartat 2.2).


% Proposal that fixes the stated problem.

\section{The Idea}

\subsection{Brief description}
\label{sec:description}

My Bachelor Degree Thesis consists on building a platform that addresses the
first problem described in section~\ref{sec:problem}. In particular, this
platform is able to:

\begin{enumerate}
  \itemsep0em
  \item Fetch and process data in realtime from any given city.
  \item Provide an easy way to extend it.
  \item Wrap the iCity platform, providing rich services instead of raw data.
\end{enumerate}

Thanks to the iCity platform, this platform is already able to respond to a
wide variety of data types. Some examples being:

\mylist
  \item Air pollution.
  \item Traffic.
  \item Irrigation control.
  \item Pedestrian flow.
\mylistend

At first I thought that I could also use data from the {\it
OpenDataBCN}\footnote{http://opendata.bcn.cat/opendata/en/} initiative.
However, I later discovered that this initiative only provides static content,
so I had to drop this idea.

Moreover, this platform has been designed to be as modular and agnostic as
possible. One of the consequences of this is that we could ideally integrate
more platforms (apart from the already existing iCity platform) without too much
trouble. This is not something that I have deeply researched, but it should be
doable.

\subsection{Wrapping the iCity Platform}

One of the main points of this project is that I am going to wrap the API of
the iCity Platform with endpoints of my own that will provide rich information
instead of raw data. This is really important because:

\begin{enumerate}
  \itemsep0em
  \item As a platform, we do not have to worry about the {\bf wide variety} of
data types, because the iCity platform is already abstracting away this issue.
  \item It gives more freedom to this {\bf platform}. As I pointed out in
section~\ref{sec:description}, this platform is not hardly tied to the iCity
platform. Therefore, even in the unlikely event that iCity gets deprecated or
dies, this platform can still fetch the data from somewhere else without too
much trouble. Of course, this other platform has to have the same guarantees as
iCity. On the other hand, if iCity is up and running, we could even consider
adding more sources of data to this platform without too much trouble.
  \item It gives the {\bf end developer} more freedom. This platform does not
replace iCity in any way. Therefore, developers can target the iCity platform
and this platform at the same time if they really want to. However, using this
platform should be enough.
  \item It is {\bf reliable}. The iCity platform is backed by the \ac{EU}. This
means that we can feel safe when using this platform.
\end{enumerate}

\subsection{Realtime}

This is the core concept behind this project. The real deal here is that all
the fetched data has to be processed in realtime. This requirement comes from
the way that the majority of targeted sensors work. Let's consider that we want
to track the levels of pollution of the air of our city. The levels of
pollution might vary during the day, and we might want to study how are these
variations occurring and how can we decrease the levels of air pollution from
this observation. Therefore, for this case we need to be tracking the levels of
pollution through the entire day. In this simple case, we realize that
processing this data in realtime is the only reliable solution to this.

In order to address this major neeed I have chosen the Storm framework. This
framework is the very base of this platform and it is thanks to this framework
that this platform can be even considered in the first place. I do not
want to get into many details in this section, but if you want to read more
about why I chose Storm, you might want to read the section~\ref{sec:state} of
this memoir.


% Showing some similar projects.

\subsection{TODO: Similar projects}

Després parlaré de coses com projectes similars (apartat 2.4) que existeixen,
alternatives, etc.


% Finally, talk about the challenges that I've been through while developing
% this project.

\section{TODO: Challenges and limitations}

També parlaré dels reptes que m'he trobat creant aquesta aplicació (apartat
2.5).


\section{Conclusions}

Aquesta és la secció típica de conclusions. Com que encara estic a mitges, no
sé si això serà molt llarg o que. Si és molt llarg, el més segur és que acabi
dividint-ho en més subseccions. Ja veurem. (+ el que he après en aquesta
universitat)

\section{Bibliography}

Finalment ficaré l'apartat de bibliografia del projecte.

\end{document}
