 
\section{The core infrastructure}
\label{sec:implementation_core}

In this section and in the following sections \ref{sec:implementation_aqs} and
\ref{sec:implementation_bsp}, I will be talking about the Storm application.
I have named the Storm application ``Snacker''. That is why all the packages
name start by ``snacker''.

\subsubsection*{SBT}

The Storm application is built with the SBT\footnote{http://www.scala-sbt.org/}
toolchain. This toolchain is the {\it de facto} standard for Scala software and
it is similar to Java's Maven or Ant. The configuration for SBT resides inside
the ``project/Build.scala'' file. This file does the following:

\begin{enumerate}
  \itemsep0em
  \item It sets up basic information about this project.
  \item It tells SBT which version of Scala has to be used.
  \item It defines the different modules of this application.
  \item It tells SBT the dependencies for each module.
\end{enumerate}

SBT is the best toolchain that I have found to build Scala projects, and this
choice has affected on how the code is structured.

\subsubsection*{The main function and the core package}

The core infrastructure is divided into two Scala packages:

\begin{enumerate}
  \itemsep0em
  \item The package {\bf com.mssola.snacker}. This package contains the main
function and it is located in the ``src'' directory. This main function is
responsible for:
    \mylist
      \item Initializing the application.
      \item Loading the services.
      \item Executing the services.
    \mylistend
  \item The package {\bf com.mssola.snacker.core}. This package contains a set
of classes, objects and traits that will be used by the services. It is
located in the ``snacker-core'' directory. We can think of this package as the
main {\it library} for the services that are built on top of Storm.
\end{enumerate}

The most important thing about the com.mssola.snacker.core package is the {\bf
BaseComponent} object. This object is expected to be subclassed by each of the
services to be loaded. It is important because this object is responsible for
initializing each service and setting it up.

