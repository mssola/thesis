
\section{The Problem}
\label{sec:problem}

As mankind moves forward, new technologies appear and, with them, new
opportunities. This situation has been beneficial in any sector imaginable,
especially in the cities.

Cities around the globe have looked at new technologies as a way to improve the
quality of life of their citizens. First and foremost, cities that are willing
to improve themselves have to answer questions such as: what do the citizens
need ? What can be improved ? Is this city efficient enough with the current
resources ?

A way to answer to these questions is simply to create a net of sensors that
will generate data. Therefore, these cities have ended up having a huge amount
of sensors scattered all around their streets. This results in cities having to
handle an unreasonable amount of {\it raw data}.

And not only that: the kind of the fetched data can vary greatly. Therefore,
these cities end up having a {\it set} of big flows of data. This chaotic
situation can be a real headache for cities willing to innovate. Thus, we have
two main problems:

\begin{enumerate}
  \itemsep0em
  \item The large amount of realtime data to be processed.
  \item The variety of data types to be handled.
\end{enumerate}

Luckily, the \ac{EU} has acknowledged the second
problem. In order to fix this second problem for any european city, the EU has
built the {\bf iCity} Platform\footnote{http://icityproject.eu/}. This platform
builds a basic infrastructure and a set of rules in which any european city can
integrate their services into this joint effort. Right now four cities have
integrated into this platform: Barcelona, London, Genoa and Bologna. Even if
this is a rather small list of european cities, more are coming:
Rome, Helsinki, etc. For this reason, we can be sure that this project is alive
and ready to take all over Europe. Because of this, we can perfectly say that
our second problem is currently being addressed and it is just a matter of time
to get it fixed.

This is great news, but it does not really fix our first problem: cities still
have to figure out how to compute all this data being fetched on realtime and
transform it into valuable information. A valid solution to this problem has to
address the following points:

\begin{enumerate}
  \itemsep0em
  \item It has to be compatible with the iCity \ac{API} for practical reasons.
  \item It has to process data in realtime. The vast majority of sensors
generate data on realtime, so it is important to get this information also in
realtime.
  \item It has to be open source. This way cities can rely on this platform
without getting tied into any private company. Furthermore, this would expand
the possibilities for this platform for any given city.
\end{enumerate}
