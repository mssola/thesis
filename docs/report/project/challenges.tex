
\section{Challenges and limitations}

\subsection{Challenges}

When any person starts a project, he has to have clear that, regardless of the
topic, any project will come with its challenges. This has been true for my
Thesis, and I think it is important to describe the challenges that I have
faced.

It has passed some months now since I started this project. When I started I
hoped to obtain all the information also from the Barcelona OpenData
iniciative. However, this iniciative turned out to be not what I expected: it
is just a repository of static data. Since my platform requires dynamic and
realtime data, fetching static data is pointless. This situation did not
affect this project greatly, since the iCity platform turned out to be exactly
what I needed, and, therefore, I did not care about Barcelona's OpenData any
longer.

Another challenge has been that I am doing this project with the Scala
programming language. To be honest, I could have done this project with just
Java, since Storm is available to any programming language that runs on the
JVM. Scala, however, is more modern than Java, and I have had more fun writing
Scala code than Java. So, even if I am new to Scala and it is a challenge,
it is a challenge in which I am fully motivated.

Hardware has proven to be a challenge too. I have never done benchmarks
regarding different computers, or given any specifications on the hardware that
a program of mine requires. So it has been definitely a new experience, and of
course a challenge.

Last but not least, a final challenge has been this report itself. First of
all, it has been a challenge because I have never had to write down a report
like this: extense, technical and detailed. Finally, it has been a challenge
because of the language. When you are writing a document like this in a
language other than your mother tongue, it can be tricky.

\subsection{Limitations}

In this sections I want to state the limitations of this project. This is
useful to clarify some points.

This project does not intend to cover as much as possible in regards to the
technologies being used. This is not a project about Storm or any
other technology in particular that I might have used. Instead, this is a
project focused on solving a specific problem.

When I started this project I even considered the idea of {\it designing} my
own set of sensors. This is of course unrealistic, so I sticked with the iCity
platform. As I have stated previously, I could have added support for more
platforms than iCity, but it would have been impossible for time constraints.
Anyways, I limited this project to be {\it iCity-specific}. This does not mean
that this platform cannot run with other platforms.
