
\section{The Idea}

\subsection{Brief description}
\label{sec:description}

My Bachelor Degree Thesis consists on building a platform that addresses the
first problem described in section~\ref{sec:problem}. In particular, this
platform is able to:

\begin{enumerate}
  \itemsep0em
  \item Fetch and process data in realtime from any given city.
  \item Provide an easy way to extend it.
  \item Wrap the iCity platform, providing rich services instead of raw data.
\end{enumerate}

Thanks to the iCity platform, this platform is already able to respond to a
wide variety of data types. Some examples being:

\mylist
  \item Air pollution.
  \item Traffic.
  \item Irrigation control.
  \item Pedestrian flow.
\mylistend

At first I thought that I could also use data from the {\it
OpenDataBCN}\footnote{http://opendata.bcn.cat/opendata/en/} initiative.
However, I later discovered that this initiative only provided static content,
so I had to drop this idea.

Moreover, this platform has been designed to be as modular and agnostic as
possible. One of the consequences of this is that we could ideally integrate
more platforms (apart from the already existing iCity platform) without too much
trouble. This is not something that I have deeply researched, but it should be
doable.

\subsection{Wrapping the iCity Platform}

One of the main points of this project is that I am going to wrap the API of
the iCity Platform with endpoints of my own that will provide rich information
instead of raw data. This is really important because:

\begin{enumerate}
  \itemsep0em
  \item We, as a platform, do not have to worry about the {\bf wide variety} of
data types, because the iCity platform is already abstracting away this issue.
  \item It gives more freedom to this {\bf platform}. As I pointed out in
section~\ref{sec:description}, this platform is not hardly tied to the iCity
platform. Therefore, even in the unlikely event that iCity gets deprecated or
dies, this platform can still fetch the data from somewhere else without too
much trouble. Of course, this other platform has to have the same guarantees as
iCity. On the other hand, if iCity is up and running, we could even consider
adding more sources of data to this platform without too much trouble.
  \item It gives the {\bf end developer} more freedom. This platform does not
replace iCity in any way. Therefore, developers can target the iCity platform
and this platform at the same time if they really want to. However, using this
platform should be enough.
  \item It is {\bf reliable}. The iCity platform is backed by the EU. This
means that we can feel safe when using this platform.
\end{enumerate}

\subsection{Realtime}

This is the core concept behind this project. The real deal here is that all
the fetched data has to be processed in realtime. This requirement comes from
the way that the majority of targeted sensors work. Let's consider that we want
to track the levels of pollution of the air of our city. The levels of
pollution might vary during the day, and we might want to study how are these
variations occurring and how can we decrease the levels of air pollution from
this observation. Therefore, for this case we need to be tracking the levels of
pollution through the entire day. In this simple case, we realize that
processing this data in realtime is the only reliable solution to this.

In order to address this major neeed I have chosen the Storm framework. This
framework is the very base of this platform and it is thanks to this framework
that this platform can be even considered in the first place. I do not
want to get into many details in this section, but if you want to read more
about why I chose Storm, you might want to read the section~\ref{sec:state} of
this memoir.
