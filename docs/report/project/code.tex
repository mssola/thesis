
\section{Source Code}

This project is open source. Therefore it can be retrieved, modified, read,
etc. by anyone that is interested in it. I have developed this project with an
open source approach because:

\begin{enumerate}
  \itemsep0em
  \item I believe that any project developed under an {\bf academic} course
should be open source. In the end, one of the main goals of a thesis is to
learn. If this project can help other students or teachers to have a deeper
understanding of the topics explained here, the better.
  \item It can be created a {\bf business model} around this platform that
focusses on the services it can provide, instead of selling licenses. I have
not focussed on business models in my thesis, but I can see multiple way to
monetize this platform. In these models the fact that the platform is open
source does not make any harm to them.
  \item The code for this platform can be {\bf useful} for other people with
similar projects. Even it might seem a quite altruistic way of thinking (and it
is), there are also selfish reasons to do this: if other people use this, it is
likely that they will contribute back and thus making the platform better.
Therefore, I believe that making this platform open source will increase the
chances for it to be as successful as possible.
  \item It is the {\bf fair} thing to do. This is more a personal believe, but
to me it is important. I have learnt a lot with open source software, and that
is the main reason why I think it is the fairest thing to do: so others can
have the same opportunities that I had.
\end{enumerate}

The license for this project is the GPL version 3. You will find a copy of this
license in the root directory of the project. In order to retrieve the project,
you will need {\bf Git}\footnote{http://git-scm.com}. Git is the version
control system that I have used to develop this project. Now, with git
installed you have to execute the following command:

\begin{center}
  git clone https://github.com/mssola/thesis
\end{center}

With this command you will download the project in the current directory. It is
extremely recommended to read the ``README.md'' file before going any further.
In this document I have carefully explained have a developer can:

\mylist
  \item Download and install all the dependencies.
  \item Build the project.
  \item Run the platform.
\mylistend

There are also ``README.md'' files inside some of the subdirectories. This
files contain more information for the given subdirectory. It is also
adviceable to read the contents of these files in order to have a better
understanding of the project.
