 
\section{Used Technologies}

\subsection{Linux}

This platform targets Linux. It is the Operating System that I have used to
develop this platform and the target of this platform. This is because, for a
variety of reasons, the vast majority of Internet servers use Linux as their
Operating system. Therefore, it makes perfect sense to use Linux as the target
operating system.

Note that this platform should also work in MacOS X and other BSD variants.
However, I have not tested this platform in these operating systems, so I
cannot claim anything in regards to availability here.

\subsection{Java \& Scala}

Storm is implemented with Java and Clojure. This two languages sit in top of
the JVM. The JVM is a virtual machine that can be used by any language that
knows how to produce bytecode for it. This includes languages such as: Java,
Clojure, Scala, Groovy, etc.

Moreover, all the languages on top of the JVM can share packages. That is, a
JAR file can be used as a library from any of these languages, regardless of
the language that was used in the library inside the JAR file. This means that
languages like Scala can re-use the huge list of Java packages that the
community has implemented.

This means that in order to implement this platform I could have used any
language that sits on top of the JVM without any real problem. I have chosen
Scala for this because:

\mylist
  \item It is as {\bf fast} as Java, so there is no performance penalties
because of the language if we compare it with Java.
  \item It is {\bf modern}. Scala is a more recent language. This means that it
has had the influence of languages that are more recent than Java. This results
in Scala having many concepts from functional programming languages, concepts
from Python, Ruby, etc. It is an absolute pleasure to write Scala code.
  \item It has a {\bf robust} approach of concurrency. It is far more intuitive
than Java's lock/unlock mechanisms.
  \item It is {\bf stable}. Even if it is more recent than Java, Scala is rock
solid. As an example, big companies like Twitter and Foursquare have lots of
Scala code running on production.
\mylistend

\subsection{Storm}

As I have said in the section \ref{sec:state_storm}, the technology that I am
using to process all the data is Storm.

\subsection{Cassandra}
\label{sec:cassandra}

I do not store a lot of data in this platform, but I do store some data like
the state of the cluster. To achieve this I use Apache Cassandra. We could
have chosen any DataBase Management System (DBMS, from now on) here to do the
job, but I have chosen Cassandra for the following reasons:

\mylist
  \item It has MapReduce support. This is the main reason that I have chosen
Cassandra instead any other traditional DBMS like PostgreSQL.
  \item It is {\bf fault-tolerant}, and supports several replication policies
across multiple clusters.
  \item It is {\bf descentralized}. There is no single point of failure.
  \item Its NoSQL nature has been helpful throughout the development of this
platform. Moreover, its query language (CQL) is quite similar to standard SQL,
so there has not been any major learning curve for me to use Cassandra.
\mylistend

\subsection{Go}
\label{sec:go}

To do.
