% Copyright (C) 2014 Miquel Sabaté Solà <mikisabate@gmail.com>
%
% This program is free software: you can redistribute it and/or modify
% it under the terms of the GNU General Public License as published by
% the Free Software Foundation, either version 3 of the License, or
% (at your option) any later version.
%
% This program is distributed in the hope that it will be useful,
% but WITHOUT ANY WARRANTY; without even the implied warranty of
% MERCHANTABILITY or FITNESS FOR A PARTICULAR PURPOSE.  See the
% GNU General Public License for more details.
%
% You should have received a copy of the GNU General Public License
% along with this program.  If not, see <http://www.gnu.org/licenses/>.

\section{Project scope}

\subsection{The context: Big data and streaming}

Once upon a time, Google released a paper called: ``MapReduce: Simplified Data
Processing on Large Clusters''\cite{mapreduce}. This was the beginning of
a revolution in the data processing front. After that, lots of efforts
have been put in this direction, being Hadoop one of the most important
accomplishments. Hadoop (and related technologies) has made it possible
to store and process data at scales previously unthinkable.

However, these technologies have one major drawback: they are not realtime
systems, nor are they meant to be. In order to solve this issue one has to
manually implement a network of queues and workers. These workers would
eventually send messages, update databases, and send new messages to other
queues for further processing. This, of course, has some serious limitations:
it's tedious, it doesn't scale, and it has little fault-tolerance.

In order to fix the previously cited issues from the core, some projects like
Yahoo! S4 and Twitter Storm emerged. Both projects have some differences but
they have a clear focus: to ease the writing of parallel realtime computation.
Sadly, nowadays the S4 project is dead. This is why in this thesis I'm picking
Storm\cite{storm} as the base technology (plus, the Storm project has a very
active community).

\subsection{My proposal}

My Bachelor Degree Thesis is about building an infrastructure that is capable
of providing a set of services from the data that has been collected and
processed. The raw data for this infrastructure will come from OpenData
BCN\cite{opendata} iniciative. That is, the infrastructure that I'm going
to build for my thesis will provide additional features to the OpenData
BCN iniciative that are only possible by building a cluster that will
process data continiously.

This infrastructure will fetch and process all this data in realtime, using
Storm as the base technology in the software front. This is important because
the users of all the services build upon these infrastructure will be able to
fetch all the processed data also in realtime.

This means that I am not just going to do research only in the Big data and
streaming front, but I'm also going to research more deeply in the whole concept
of ``Smart City''. This is because this infrastructure will be built in a way
so other services can fetch our processed data to improve, promote and have a
better understanding of the city of Barcelona.

I am very motivated for this project for the following reasons:

\mylist
  \item {\bf Research}. Because of this thesis, I will be doing a lot of
research. My main focus will be: Big data, streaming processing and smart
cities. I will be investigating, for each of them, their state of the art, what
should we know about and what can we do with them.
  \item {\bf Learn} some technologies as a side effect. My base technology will
be the Storm framework. This means that I will be learning and improving my
skills in programming languages like Java and Scala. I will also be improving my
skills with concurrent programming. I will further also my knowledge on hardware
by analyzing what kind of machines we need in order to provide and scale all the
services.
  \item {\bf Provide} a useful technology base. The main goal of this Thesis is
to create a base technology that will be useful to other people. This way we
will be able to provide a set of services that will potentially improve the
city of Barcelona on multiple ways. I am really optimistic about this project,
and I can see other thesis from other students about improving or extending the
infrastructure that I'll be building in my thesis.
  \item {\bf Practice}. The Bachelor Degree Thesis is the final stop of this
journey that I've been doing in the Barcelona School of Informatics. I've
learnt a lot and I am ready to apply some of the concepts learnt in this
university to this Thesis.
\mylistend

\subsection{Goals}

In the previous section I have clarified my proposal and my motivations. Now
it is time to specify the specific goals of this project:

\mylist
  \item Design a cluster that can match our requirements.
  \item Design a base system that will fetch and process all the data.
  \item Provide a couple of useful services on top of the base system.
\mylistend

\subsection{Limitations and risks}

There are more things to evaluate before starting the project. In this section
I will describe the limitations and the risks of this project.

I will start by specifying the limitations. This project does not intend to
cover as much as possible in regards to the technologies that I will be using.
This is not a project about Storm or any other technology that I will be using.
Instead, this is a project focused on solving a specific problem.

Another limitation comes with the hardware. First of all, the sensors. An ideal
situation would be that I design the sensors and I establish a way for the
sensors to communicate with the cluster. However, this is unlikely to happen,
so that I am relying in an already existing solution such as OpenData BCN.
Another limitation is the cluster itself. Ideally I would design a base cluster
to handle all the software, but it is extremely unlikely that this cluster
gets built anytime soon. For this reason, I will do my experiments in my local
machine and, hopefully, also in the clusters provided by the Computer
Architecture Department.

There are some risks to evaluate too. First of all, I want to talk about time
related problems. If I do not have enough time to build the software, I intend
to cut on services. That is, I won't be building more services. The core
software is a must, though, so there won't be any time related issues here.

Another possible risk comes with the possibility that maybe the OpenData BCN
and the iCity initiatives don't meet my expectations. In this case I will be
developing an ``alternate'' solution, by specifying how ideally the sensors or
the interface to the sensor would be sending requests to my cluster.

\subsection{About the development}

Lastly, I want to clarify how I want to develop this project. First of all, the
code that will make all this to happen is open source. This means that all of
the base technology and the code that I will be developing is all open source.
In fact, all my code is hosted on
Github\footnote{https://github.com/mssola/thesis}. This is really important for
me for the following reasons:

\espai

\mylist
  \item I am a strong advocate of Open Source.
  \item It could help future thesis based on this infrastructure.
  \item It could help anyone in the world that is developing something similar.
  \item It is fair.
\mylistend

My workflow will consist in discussing any doubts that I would have with my
teacher and keep on developing the whole platform in an open way. Finally, I
want to point out my resources, that I quite humble. I will spend most of my
time doing this project, with all the resources that I have.
